%%%%%%%%%%%%%%%%%%%%%%%%%%%%%%%%%%%%%%%%%%%%%%%%%%%%%%%%%%%%%%%%%%%%%%%%%%%%%%%
%%%%%%%%%%%%%%%%%%%%%%%%%%%%%%%%%%%%%%%%%%%%%%%%%%%%%%%%%%%%%%%%%%%%%%%%%%%%%%%
%%%%%%%%%%%%%%%%%%%%%%%%%%%%%%%%%%%%%%%%%%%%%%%%%%%%%%%%%%%%%%%%%%%%%%%%%%%%%%%

%% minimal working example pro diagram s diakritikou a matematickou notací
%% v popiscích

%% typeset by TeX by Lubomir Stepanek

\documentclass{article}
\usepackage[T1]{fontenc}
\usepackage[utf8]{inputenc}
%\usepackage{amsmath, amsfonts, mathtools}
%\usepackage{graphicx}
%\usepackage{eso-pic}
%\usepackage[czech]{babel}
%\usepackage{epstopdf}    % důležité pro zrenderování .eps ve Windows !!!
%\usepackage{float}
%\usepackage[bottom]{footmisc}
%\usepackage[margin = 2.5cm]{geometry}

%% ----------------------------------------------------------------------------

\title{%
  Tvorba diagramů\\
  s českými popisky\\
  a\\
  matematickou notací\\
  v prostředí \textsf{R}
}
\author{Lubomír Štěpánek}
\date{\textit{14. srpna 2018}}


%% ----------------------------------------------------------------------------

%% vkládám své vlastní styly, knihovny, některé globální parametry ------------
\input{my_styles.tex}


%% ----------------------------------------------------------------------------

\begin{document}

\maketitle         % nadpis

\tableofcontents   % obsah

\AddToShipoutPictureBG*{%
  \AtPageLowerLeft{%
    \hspace{\paperwidth}%
    \raisebox{\baselineskip}{%
      \makebox[0pt][r]{
        {\footnotesize Vysázeno pomocí \textsf{R}-ka a \TeX-u\quad}
      }
}}}


\section{Ukázkový příklad}

Uvažujme dvě nebiasované laplaceovské kostky vždy o počtu ok 1 až 6,
kterými současně hodíme. Hledejme pravděpodobnost, že součet ok na obou
kostkách bude roven číslu $x \in \mathbb{R}$.

Součet ok, která padnou na obou kostkách, vnímejme jako náhodnou veličinu
a označme ji $X$. Hledáme tedy $P(X = x)$. Po troše uvažování a základní
aritmetiky můžeme odvodit,
že pravděpodobnostní funkce náhodné veličiny $X$ je $P_{X}(x) = P(X = x)$
rovna

\begin{equation*}
  P_{X}(x) = P(X = x) = \left\{
  \begin{array}{ll}
  \frac{6 - \lvert x - 7 \rvert}{36}, & x \in \{2, 3, \ldots, 12\} \\
  \phantom{iiiiiiii}0, & \text{jinak}.
  \end{array} \right.
\end{equation*}

Na obrázku~\ref{pravdepodobnost_souctu_x} je pak grafická prezentace
pravděpodobnostní funkce $P_{X}(x)$ pro rozumné hodnoty $x$.

\begin{figure}[H]
  \centering
  \includegraphics[width = 1.0\hsize]{%
    pravdepodobnost_souctu_x.eps%
  }
  \caption{%
    Pravděpodobnosti $P(\textrm{součet ok obou kostek} = x)$ pro
    jednotlivé hodnoty $x$, kterých může součet
    ok dvou hozených kostek nabýt.
    \label{pravdepodobnost_souctu_x}%
  }
\end{figure}

Na obrázku~\ref{pravdepodobnost_souctu_x_formalneji} pak tatáž
pravděpodobnostní funkce $P_{X}(x)$ pro rozumné hodnoty $x$, avšak
s formálnějším popiskem svislé osy.

\begin{figure}[H]
  \centering
  \includegraphics[width = 1.0\hsize]{%
    pravdepodobnost_souctu_x_formalneji.eps%
  }
  \caption{%
    Pravděpodobnostní funkce $P_{X}(x) = P(X = x)$, kde
    \mbox{$x \in \{1, 2, 3, \ldots, 12 \}$}, pro náhodnou veličinu
    $X$ odpovídající součtu ok dvou hozených laplaceovských kostek.
    \label{pravdepodobnost_souctu_x_formalneji}%
  }
\end{figure}

Do třetice a pro cvičné účely je stejný diagram i na
obrázku~\ref{pravdepodobnost_souctu_x_neformalne}, kde je v~popiscích
os zcela upuštěno od matematické notace\footnote{Všechny tři
diagramy byly vytvořeny pouze s využitím jazyka a prostředí \textsf{R}.}.
Popisky však obsahují bohatou českou diakritiku.

\begin{figure}[H]
  \centering
  \includegraphics[width = 1.0\hsize]{%
    pravdepodobnost_souctu_x_neformalne.eps%
  }
  \caption{%
    Pravděpodobnosti jednotlivých hodnot, kterých může součet
    ok obou hozených kostek nabýt.
    \label{pravdepodobnost_souctu_x_neformalne}%
  }
\end{figure}

\section{Kód pro vytvoření všech tří diagramů v \textsf{R}}

Nakonec se ještě podívejme, jaký kód v prostředí \textsf{R} takové diagramy
vytvoří.

\begin{lstlisting}[style = custom_R]
# ----------------------------------------------------------------------------------------------------------------------

#\# tabulka s pravděpodobnostmi, že při hodu dvěma
#\# laplaceovými kostkami bude součet ok obou
#\# kostek roven číslu $x$

my_table <- list(
    "x" = as.numeric(1:12),
    "pravdepodobnost_x" = unlist(
        lapply(
            1:12,
            function(x){
                sum(outer(1:6, 1:6, "+") == x) / 36
            }
        )
    )
)


# ----------------------------------------------------------------------------------------------------------------------

#\# diagram součet $x$ vs. pravděpodobnost, že obě hozené kostky
#\# mají součet ok roven číslu $x$

cairo_ps(
    file = "pravdepodobnost_souctu_x.eps",
    width = 8,
    height = 5.4,
    pointsize = 16,
    family = "serif"     # patkový font kvůli matematické notaci
)

par(mar = c(4.1, 4.1, 0.1, 0.1))

plot(
    x = my_table[["x"]],
    y = my_table[["pravdepodobnost_x"]],
    xlab = expression(italic(x)),
    ylab = "",
    xaxt = "n",    # zakazuji vykreslení vodorovné osy
    yaxt = "n",    # zakazuji vykreslení svislé osy
    pch = 19
)

# všimněme si, že má-li popisek osy obsahovat současně písmeno
# (či písmena) s diakritikou (ě ve slově pravděpodobnost)
# a matematickou notaci ($P(\textrm{součet ok obou kostek} = x)$),
# není možné použít obvyklý xlab či ylab, ale popisek se musí
# sestavit pomocí funkce pro text kopírující osu diagramu mtext();
# ani engine cairo\_pdf() či cairo\_ps() oboje najednou neumí

mtext(
    text = "pravděpodobnost     (součet ok obou kostek =   )",
    side = 2,
    line = 3
)   # počet mezer jsem prostě jen odhadl metodou error-trial,
    # celkem "fishy", není to moc elegantní a zabere to čas,
    # ale dle mě a mnohých fór asi jediné řešení pro
    # češtinu \& matematiku současně v diagramech v R

mtext(
    text = expression(italic(P)),
    side = 2,      # side = 2 pro svislou osu
    line = 3,      # vzdálenost od canvasu diagramu
                   # v jednotkách vodorovné osy
    at = 0.0655    # opět, pozici dle jednotky svislé
                   # osy odhaduji od oka
)

mtext(
    text = expression(italic(x)),
    side = 2,
    line = 3,
    at = 0.168     # same story here
)

axis(
    side = 1,
    at = 1:12,
    labels = as.character(1:12)
)

axis(
    side = 2,
    at = seq(0, 0.15, 0.05),
    labels = gsub(
        "\\.",
        ",",
        format(
            round(
                seq(0, 0.15, 0.05),
                digits = 2
            ),
            nsmall = 2
        )
    )   # nahrazuju desetinnou tečku za čárku, must-do v česky
        # mluvících diagramech
)

dev.off()


# ----------------------------------------------------------------------------------------------------------------------

#\# stejný diagram, ale pro cvičné účely tentokrát v popiscích
#\# použijeme jen matematickou notaci, tvorba popisku se tím
#\# zjednoduší

cairo_ps(
    file = "pravdepodobnost_souctu_x_formalneji.eps",
    width = 8,
    height = 5.4,
    pointsize = 16,
    family = "serif"     # patkový font kvůli matematické notaci
)

par(mar = c(4.1, 4.1, 0.1, 0.1))

plot(
    x = my_table[["x"]],
    y = my_table[["pravdepodobnost_x"]],
    xlab = expression(italic(x)),
    ylab = expression(
        paste(
            italic(P)[italic(X)],
            "(",
            italic(x),
            ") = ",
            italic(P),
            "(",
            italic(X),
            " = ",
            italic(x),
            ")",
            sep = ""
        )
    ),
    xaxt = "n",    # zakazuji vykreslení vodorovné osy
    yaxt = "n",    # zakazuji vykreslení svislé osy
    pch = 19
)

axis(
    side = 1,
    at = 1:12,
    labels = as.character(1:12)
)

axis(
    side = 2,
    at = seq(0, 0.15, 0.05),
    labels = gsub(
        "\\.",
        ",",
        format(
            round(
                seq(0, 0.15, 0.05),
                digits = 2
            ),
            nsmall = 2
        )
    )   # nahrazuju desetinnou tečku za čárku, must-do v česky
        # mluvících diagramech
)

dev.off()


# ----------------------------------------------------------------------------------------------------------------------

#\# do třetice stejný diagram, ale bez matematické notace
#\# v popiscích

cairo_ps(
    file = "pravdepodobnost_souctu_x_neformalne.eps",
    width = 8,
    height = 5.4,
    pointsize = 16,
    family = "serif"     # patkový font kvůli matematické notaci
)

par(mar = c(4.1, 4.1, 0.1, 0.1))

plot(
    x = my_table[["x"]],
    y = my_table[["pravdepodobnost_x"]],
    xlab = "možná hodnota součtu ok obou kostek",
    ylab = "pravděpodobnost dané hodnoty součtu ok kostek",
    xaxt = "n",    # zakazuji vykreslení vodorovné osy
    yaxt = "n",    # zakazuji vykreslení svislé osy
    pch = 19
)

axis(
    side = 1,
    at = 1:12,
    labels = as.character(1:12)
)

axis(
    side = 2,
    at = seq(0, 0.15, 0.05),
    labels = gsub(
        "\\.",
        ",",
        format(
            round(
                seq(0, 0.15, 0.05),
                digits = 2
            ),
            nsmall = 2
        )
    )   # nahrazuju desetinnou tečku za čárku, must-do v česky
        # mluvících diagramech
)

dev.off()

# ----------------------------------------------------------------------------------------------------------------------
\end{lstlisting}

\section{Závěr}

Základním ideou tvorby diagramů s českými popisky je použití
enkapsulovaného postscriptu (\texttt{.eps}) a (nejen) \textsf{R}-kového
enginu \texttt{cairo}, který narozdíl od běžných základních enginů
(\texttt{pdf}, \ldots)
pro sázení vektorových obrázků používá znakovou sadu extended ASCII,
tedy minimálně 512 znaků, což pokryje i česká písmena s~diakritikou.
V \textsf{R} jde o funkci \texttt{cairo\_ps()}, která spustí vývojářský
mód, vytvoří podle kódu diagram, zrenderuje ho a nakonec uloží ve formátu
enkapsulovaného postscriptu (přípona \texttt{.eps}).
Obdobná funkce \texttt{cairo\_pdf()} vrátí koncový formát \texttt{.pdf}.
Obě funkce jsou součástí balíčku \texttt{grDevices}, který se inicializuje
automaticky po spuštění \textsf{R}, takže není nutné nic navíc instalovat.
Obdobné funkce a~možná trochu lepší uživatelské možnosti nabízí
\textsf{R}-kový balíček \texttt{Cairo}, nicméně pro uvedené tři diagramy
postačila jen základní funkce \texttt{cairo\_ps()}.

Přestože není dobrou praxí vkládat do \TeX-ového dokumentu obrázky
jako \texttt{.pdf}, reálně by mělo vše fungovat i se zachováním
české diakritiky. Formát postscriptu je preferován proto, že jde
ve skutečnosti o vlastní jazyk (takže obrázek s příponou \texttt{.eps}
nelze otevřít v běžném editoru obrázků, ale naopak lze otevřít
v textovém editoru). Při znalosti kódu postscriptu (je podobný jazyku
PROLOG) lze dokonce tímto způsobem dělat postprocessingové
úpravy obrázků pouhou úpravou jeho kódu (bez nutnosti obrázek znovu
generovat).

Kombinování českých písmen s diakritikou a matematické notace v jednom
popisku je svízelné, ale možné. Jedním z řešení\footnote{Osobně žádné jiné
bohužel neznám a po dlouhém (řádově dny v roce 2017) pátrání po všemožných
Q\&A fórech ani nic nenasvědčuje, že by jiné existovalo.} je
tvorba takového popisku "ve dvou vrstvách", postupně, tedy nejdříve
vytvoření textového podkladu -- ten může obsahovat diakritiku,
a poté až umístění matematické notace obalené \mbox{v~\textsf{R}-kové}
funkci \texttt{expression()}. Obě "vrstvy" popisku lze vytvořit pomocí
\textsf{R}-kové funkce \texttt{mtext()}. Naopak známé grafické argumenty
\texttt{xlab}, \texttt{ylab} samy o~sobě v~tomto duchu fungovat nebudou
(engine \texttt{cairo} pravděpodobně neumí zpracovat výstup jednoho
generického grafického příkazu v \textsf{R} tak, aby zrenderoval
současně diakritiku i~matematickou notaci, a v ten okamžik přestane
kód diagramu dále renderovat, aniž by vrátil chybovou hlášku -- výsledkem
je tak "diagram" s chybějícími komponentami). Příkladem kombinace
matematické notace a diakritiky v jednom popisku je
obrázek~\ref{pravdepodobnost_souctu_x}.


\end{document}


%% ----------------------------------------------------------------------------

%%%%%%%%%%%%%%%%%%%%%%%%%%%%%%%%%%%%%%%%%%%%%%%%%%%%%%%%%%%%%%%%%%%%%%%%%%%%%%%
%%%%%%%%%%%%%%%%%%%%%%%%%%%%%%%%%%%%%%%%%%%%%%%%%%%%%%%%%%%%%%%%%%%%%%%%%%%%%%%
%%%%%%%%%%%%%%%%%%%%%%%%%%%%%%%%%%%%%%%%%%%%%%%%%%%%%%%%%%%%%%%%%%%%%%%%%%%%%%%





